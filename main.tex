% !TEX program = xelatex
\documentclass{joulabreport}

% Title page information
\coursename{数值分析A}
\expname{插值法}
\classname{信嵌1221}
\author{Ilya Sutskever}
\studentid{2022123456}
\thisdate{2024/12/9}
\location{定海楼602}
\teacher{Geoffrey Hinton}

\begin{document}
\maketitle%
{\song\xiaosi% 设置字体和字号
\section{实验原理}
数值分析中的插值问题源于科学研究和工程应用中的一个基本需求\cite{ref1}:通过有限的离散观测数据重建连续函数关系。这种重建过程不仅需要满足数学上的严格性\cite{ref2},还要考虑实际应用中的计算效率和精度要求。本节将系统阐述插值问题的理论基础及其实践意义\cite{ref3,ref4}。

插值问题的数学本质是在给定一组观测数据点$(x_i, y_i)$($i=0,1,\ldots,n$)的条件下,构造一个满足特定性质的函数$P(x)$。这里的$\{x_i\}$构成互异的节点序列,而$y_i$则代表相应的观测或函数值。构造的插值函数不仅要精确通过这些数据点,还需具备良好的分析性质,以支持后续的数值计算和理论分析。

在众多插值方法中,Lagrange插值法通过其优雅的理论构造提供了一个系统的解决方案。该方法的核心在于构造一组特殊的基函数$\ell_k(x)$,定义为:
\begin{equation}
\ell_k(x) = \prod_{\substack{j=0\\j\neq k}}^n \frac{x-x_j}{x_k-x_j}
\end{equation}

这组基函数的独特之处在于其正交性质,即在任意节点$x_i$处:
\begin{equation}
\ell_k(x_i) = \begin{cases}
1, & i = k \\
0, & i \neq k
\end{cases}, \quad i,k = 0,1,\ldots,n
\end{equation}

基于这一性质,我们可以构造Lagrange插值多项式:
\begin{equation}
P_n(x) = \sum_{k=0}^n y_k\ell_k(x)
\end{equation}

对于实际应用中的误差控制,当目标函数$f(x)$在插值区间$[a,b]$上具有充分的光滑性(即具有$n+1$阶连续导数)时,插值误差可以通过如下表达式刻画:
\begin{equation}
R_n(x) = f(x) - P_n(x) = \frac{f^{(n+1)}(\xi)}{(n+1)!}\prod_{i=0}^n(x-x_i), \quad \xi \in [a,b]
\end{equation}

这一误差表达式揭示了插值精度与函数光滑性、节点分布及插值多项式次数之间的内在联系。在实际应用中,我们需要权衡这些因素:过高的插值次数可能导致龙格现象,而不合理的节点分布则可能放大误差。因此,在具体实践中,我们通常采用切比雪夫节点来优化节点分布,或在较大区间上采用分段低次插值策略,从而在保证精度的同时确保数值稳定性。

通过深入理解插值问题的理论基础,我们不仅能够准确把握其数学本质,还能够在实际应用中做出更明智的方法选择。这种理论与实践的结合,正是数值分析方法在科学计算中发挥重要作用的体现。
% 选择结束

\section{算法}
在此部分详细说明实验采用的插值算法步骤,可包含数学公式与伪代码。例如:

设已知节点与函数值为 $\{(x_i,y_i)\}_{i=0}^n$,插值多项式可表示为:
\begin{equation}
P_n(x) = \sum_{k=0}^{n} y_k \ell_k(x)
\label{eq:polynomial}
\end{equation}
其中
\begin{equation}
\ell_k(x) = \prod_{\substack{j=0\\j\neq k}}^n \frac{x - x_j}{x_k - x_j}
\label{eq:basis}
\end{equation}

% 添加伪代码实现
\begin{algorithm}[htbp]
\caption{Lagrange插值算法}
\label{alg:lagrange}
\begin{algorithmic}[1]
\Require 插值节点 $\{x_i\}_{i=0}^n$,函数值 $\{y_i\}_{i=0}^n$,待插值点 $x$
\Ensure 插值结果 $P_n(x)$
\State $P_n \gets 0$
\For{$k \gets 0$ \textbf{到} $n$}
    \State $\ell_k \gets 1$
    \For{$j \gets 0$ \textbf{到} $n$}
        \If{$j \neq k$}
            \State $\ell_k \gets \ell_k \cdot \frac{x - x_j}{x_k - x_j}$
        \EndIf
    \EndFor
    \State $P_n \gets P_n + y_k \cdot \ell_k$
\EndFor
\State \Return $P_n$
\end{algorithmic}
\end{algorithm}

% 描述算法实现步骤
\begin{enumerate}
\item 输入插值节点与对应函数值。
\item 计算每个 $\ell_k(x)$ 的分子与分母(见公式~\eqref{eq:basis})。
\item 对给定点$x$,计算所有 $\ell_k(x)$,然后求和得到 $P_n(x)$(见公式~\eqref{eq:polynomial})。
\item 输出插值结果与误差分析。
\end{enumerate}

\section{计算机程序}
此处可贴出源代码片段或简述实现步骤。例如使用Matlab或Python的代码。

\begin{lstlisting}[language=Python, caption=Lagrange插值算法实现]
def lagrange_interpolation(x_points, y_points, x):
    n = len(x_points)
    P = 0.0
    for k in range(n):
        Lk = 1.0
        for j in range(n):
            if j != k:
                Lk *= (x - x_points[j])/(x_points[k]-x_points[j])
        P += y_points[k]*Lk
    return P
\end{lstlisting}

\section{测试数据与实验结果}
在此部分展示所测试的输入数据节点及相应的插值结果。

\begin{figure}[htbp]
    \centering
    \includegraphics[width=0.75\textwidth]{figure/regression_analysis.pdf}
    \caption{插值结果与原函数对比图}
    \label{fig:regression}
\end{figure}

如图~\ref{fig:regression}所示,插值函数(红色实线)很好地拟合了原始数据点(蓝色散点)。从图中可以看出\dots

具体测试数据如下:
\begin{itemize}
\item 输入节点:$x_0=0, x_1=1, x_2=2, x_3=3$,$y_i=f(x_i)$为函数值。
\item 测试插值点:$x=1.5$,计算插值结果 $P_3(1.5)$ 并与真实值 $f(1.5)$ 对比误差。
\end{itemize}

\begin{figure}[htbp]
    \centering
    \begin{minipage}[t]{0.45\textwidth}
        \centering
        \includegraphics[width=\textwidth]{figure/minipage1.pdf}
    \end{minipage}
    \hfill
    \begin{minipage}[t]{0.45\textwidth}
        \centering
        \includegraphics[width=\textwidth]{figure/minipage2.pdf}
    \end{minipage}
    \caption{插值算法的多角度分析}
    \label{fig:analysis}
\end{figure}

如图~\ref{fig:analysis}所示,我们从多个角度分析了插值算法的性能。图~\ref{fig:mini1}展示了不同节点数下的插值效果,图~\ref{fig:mini2}显示了误差随节点数的变化趋势。

% 示例表格1:基本三线表
\begin{table}[htbp]
\centering
\caption{测试点处插值结果对比}
\label{tab:comparison}
\begin{tabular}{ccc}
\toprule[1.5pt]
$x$ & $f(x)$ (真值) & $P_3(x)$ (插值值) \\
\midrule[0.75pt]
0.0 & 1.000 & 1.000 \\
1.0 & 1.649 & 1.649 \\
1.5 & 2.250 & 2.248 \\
2.0 & 3.196 & 3.196 \\
3.0 & 6.854 & 6.854 \\
\bottomrule[1.5pt]
\end{tabular}
\end{table}

% 示例表格2:带有合并单元格和多级表头的三线表
\begin{table}[htbp]
\centering
\caption{不同节点数的插值误差分析}
\label{tab:error_analysis}
\begin{tabular}{c*{3}{c}}
\toprule[1.5pt]
\multirow{2}{*}{节点数} & \multicolumn{3}{c}{最大误差} \\
\cmidrule[0.75pt](lr){2-4}
& 均匀节点 & Chebyshev节点 & 最优节点 \\
\midrule[0.75pt]
5  & 1.24e-2 & 8.93e-3 & 7.21e-3 \\
10 & 3.56e-3 & 1.45e-3 & 1.12e-3 \\
15 & 8.92e-4 & 2.34e-4 & 1.89e-4 \\
20 & 2.45e-4 & 4.56e-5 & 3.23e-5 \\
\bottomrule[1.5pt]
\end{tabular}
\end{table}

\begin{table}[h]
\centering
\setlength{\tabcolsep}{8pt}
\renewcommand{\arraystretch}{1.2}
\caption{插值方法的多维度分析与评估}
\label{tab:method_comparison}
\begin{tabular}{
    l
    >{\raggedright\arraybackslash}p{4cm}
    >{\raggedright\arraybackslash}p{3cm}
    >{\centering\arraybackslash}p{2cm}
    >{\centering\arraybackslash}p{2cm}
}
\toprule
\textbf{方法类型} & \textbf{核心机理} & \textbf{应用场景} & \textbf{精度} & \textbf{效率} \\
\midrule

\textbf{拉格朗日} & 
\multicolumn{1}{m{4cm}}{\centering 基于基函数线性组合,全局多项式构造,高阶龙格现象} & 
\multicolumn{1}{m{3cm}}{\centering 低阶精确插值,理论分析验证} & 
高 & 
中 \\
\midrule

\textbf{牛顿法} & 
\multicolumn{1}{m{4cm}}{\centering 差商递推构造,增量式计算结构,系数复用特性} & 
\multicolumn{1}{m{3cm}}{\centering 动态节点更新,程序实现} & 
高 & 
高 \\
\midrule

\textbf{分段线性} & 
\multicolumn{1}{m{4cm}}{\centering 局部线性逼近,区间独立计算,简化数值处理} & 
\multicolumn{1}{m{3cm}}{\centering 实时计算需求,快速估值场景} & 
低 & 
高 \\
\midrule

\textbf{三次样条} & 
\multicolumn{1}{m{4cm}}{\centering 二阶导数连续,全局方程求解,最优光滑性质} & 
\multicolumn{1}{m{3cm}}{\centering 数据可视化,曲线平滑拟合} & 
高 & 
低 \\

\bottomrule
\end{tabular}
\begin{tablenotes}
\small
\item 性能指标说明:高表示性能优异,中表示性能一般,低表示性能较差
\end{tablenotes}
\end{table}

如表~\ref{tab:comparison}、表~\ref{tab:error_analysis}和表~\ref{tab:method_comparison}所示,我们提供了\dots

\section{结论}
本实验通过理论分析与数值实验深入研究了【实验主题】在【应用场景】中的应用特性。实验结果表明,该方法在处理【具体任务】时展现出了显著的优势。通过对【关键要素】的系统性分析,我们发现【核心技术/方法】能够【实现的功能】,为【更大目标】提供了可靠基础。

在实验实施过程中,我们观察到【具体实验现象】与理论预期高度吻合。特别是在【典型测试案例】中,【实验方法】给出的结果与【对照标准】的【评价指标】极小,充分验证了该方法的有效性。这种【优势特点】源于【技术原理】的基本性质——【原理解释】。然而,深入分析也揭示了该方法的局限性。随着【变量参数】的改变,虽然理论上可以【预期效果】,但会引发【负面影响】。这种现象最典型的表现是【具体问题】,即【问题表现】,影响【影响对象】的可靠性。这一问题启示我们在实际应用中需要【注意事项】。基于本次实验的发现,我们提出以下改进建议:首先,可以【改进方案1】,通过【具体措施】,从而【预期效果】;其次,考虑【改进方案2】,特别是【具体技术】,它不仅能【优势1】,还能【优势2】;最后,建议【改进方案3】,根据【依据】动态调整【调整对象】,以获得最优的【优化目标】。

这些实验成果对【理论领域】具有重要的理论意义,在【应用领域】中具有广泛的应用前景。特别是在【具体应用场景列举】等领域,本研究的方法和结论都具有直接的参考价值。

% 参考文献部分(如不需要可注释)
\begin{thebibliography}{4}
\bibitem{ref1} 李庆扬, 王能超, 易大义. 数值分析[M]. 第5版. 北京: 清华大学出版社, 2008.
\bibitem{ref2} Stoer J, Bulirsch R. Introduction to Numerical Analysis[M]. 3rd ed. New York: Springer-Verlag, 2002.
\bibitem{ref3} Berrut J P, Trefethen L N. Barycentric Lagrange Interpolation[J]. SIAM Review, 2004, 46(3): 501-517.
\bibitem{ref4} Higham N J. The numerical stability of barycentric Lagrange interpolation[J]. IMA Journal of Numerical Analysis, 2004, 24(4): 547-556.
\end{thebibliography}

} % 字体大小和字体族组结束

\end{document}
